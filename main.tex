\documentclass[12pt]{article}
\usepackage[a4paper, total={6.5in, 9in}]{geometry}
\usepackage{placeins}           % anchored tables and figures
\usepackage{enumitem}
\setlist[enumerate]{nosep}           % for numbered lists
\usepackage{longtable}          % span table over multiple pages

\RequirePackage{mathptmx} 							
%Set the font to Times New Roman. This package also alters the math-mode font to TNR.
\RequirePackage{graphicx}								
%Use pdfLaTeX.
\RequirePackage{url} % Typeset URLs properly

\title{\textbf{RBUS3904: Commonwealth Bank of Australia Case Study}}
\author{N. Khor\\
    43953970}
\date{April 09, 2020}

\begin{document}
\maketitle

This case study investigates the ethical conduct and social responsibility of the Commonwealth Bank of Australia (CBA).

%1.	Provide a brief overview of the major shareholders of Commonwealth Bank from the 2019 annual report and a brief overview of the Board of Directors. 
\section{Major Shareholders of CBA and Overview of Board of Directors}
The top twenty shareholders in hold 49.72\% of the total issued CBA shares \cite{cba19}. Six of the largest holding shareholders are described below in Table~\ref{tab:shareholders}, along with their respective industries. Under CBA’s Constitution, each ordinary shareholder who is present at the general meeting holds one vote for each share held \cite{cba19}. Therefore, shares held is representative of the voting power of that shareholder.

\FloatBarrier
\begin{table}[ht]
\centering
\caption{Top 6 CBA shareholders and their relevant industries \cite{cba19}}
\begin{tabular}{ |p{7cm}|p{4cm}|p{3.5cm}|} 
\hline
\textbf{Name of holder} & \textbf{Industry} & \textbf{Percentage of shares held (\%)} \\ 
\hline
HSBC Custody Nominees  & Banking \cite{bloomberg-hsbc} &  22.38 \\ 
\hline
J P Morgan Nominees Australia & Commercial services \cite{bloomberg-jp} & 12.79\\
\hline
Citicorp Nominees Pty Limited & Asset management \cite{bloomberg-citi} & 6.04 \\
\hline
National Nominees Limited & Financial services \cite{bloomberg-national} & 3.10 \\
\hline
BNP Paribas Noms Pty Ltd & Banking \cite{bloomberg-bnp} & 2.94 \\
\hline
Bond Street Custodians Limited & Financial services \cite{bloomberg-bond} & 0.59 \\
\hline
\end{tabular}
\label{tab:shareholders}
\end{table}
\FloatBarrier

%3.	What influence do you think the major shareholders will have on the Board? 
The majority of these large shareholders are in the banking, commercial services and financial services industries. They influence the board by deciding who become board members, through a vote. CBA’s larger shareholders may prefer board members who are trusted by those in the aforementioned industries. Shareholders participating in the same industry may also have aligned interests, opinions about directors and therefore voting behaviour.
\\

%2.	Are any Directors involved in other companies? (150 worcs, 10%)
%Accurately identifies the shareholders and Directors, and provides considerable detail of them and provides an excellent analysis of their involvement in the firm.
There are 10 members in the Board of Directors in the various committees for risk, audit, nominations, and people and renumeration. Two directors are executive, the remaining non-executive directors. Most interestingly, Shirish Apte, an Independent Non-Executive Director, previously worked in executive roles for Citi Asia \cite{cba19}. It may be purely coincidental that Citicorp Nominees Pty Ltd owns 6.04\% of the company.


\section{Breaches of Code of Conduct and Ethics}
%4.	Read Commonwealth Bank Code of Conduct and Ethics, and list as many breaches of its own code of ethics that Commonwealth has made in the past 5 years that you can find. What have been the implications of these breaches to the Commonwealth Bank (250 words, 25%)
%Answer is insightful and with a very high level of detail. Provides a solid list of breaches and implications are well researched and analysed.

The CBA Code of Conduct and Ethics claims to guide how decisions are made within the company. Most notably, it defines Value Expectations that the company is guided by. These values are "what we do is right", "we are accountable", "we are dedicated to service", "we pursue excellence", and "we get things done" \cite{code-of-conduct}. The following breaches of this Code of Conduct is described in Table~\ref{tab:breaches}.


\begin{longtable}{|p{2cm}|p{4cm}|p{4.5cm}|p{4.5cm}|} 
\caption{Examples of CBA breaching the Value Expectations}
\label{tab:breaches}
\hline
% longtable formatting of headers over multiple pages
\textbf{Value} & \textbf{Explanation of value} & \textbf{Breach examples} & \textbf{Implications of breaches} \hline \endfirsthead

\multicolumn{4}{c}
        {Table \thetable\ continued from previous page} \\
        \hline
\textbf{Value} & \textbf{Explanation of value} & \textbf{Breach examples} & \textbf{Implications of breaches}
\hline \endhead
%
\multicolumn{4}{|r|}{{Continued on next page}} \\ \hline
\endfoot
%
\endlastfoot

% Table data
What we do is right & Delivering the best possible outcome for customers with integrity and transparency & Covered up corrupt financial planners that did not deliver the good outcomes for customers \cite{barker17}. Charged customers for fees that were not provided \cite{APESB18}. & CBA incurred an extra \$470 million to provision for the fee-for-no-service scandal, including customer compensation \cite{Letts18}. CBA dropped 21 places in the Corporate Reputation Index \cite{APESB18}. \\
\hline
We are accountable  & CBA should be held accountable for their actions and follow through with their responsibilities &  Financial planning customers could not seek guidance after significant investment loss with CBA financial planners \cite{barker17}
& The Banking Royal Commission was created by the Australian Government to investigate banks such as CBA and hold them accountable for misconduct. \\ 
\hline
We are dedicated to service & Care for the customer and serve them by doing the right thing &  Incentivised unethical behaviour of financial planners by rewarding it with commissions and bonuses for high performances \cite{gordon18}. & Created a performance based sales culture for financial planners that did not align with doing right by customers. Financial planners did not act on the best interests of customers, often putting investments into high risk and high reward funds. This caused many customers to lose money \cite{gordon18}.\\
\hline
We pursue excellence & Provide a service with high standard of quality, encourage constructive feedback and learn from mistakes &  Appointed Matt Comyn as CEO, despite his possible involvement with anti-money laundering scandal. Comyn was head of retail banking for CBA, which was involved in the money laundering regulation breaches \cite{dickson18}.  &  The internal hire of a new CEO makes CBA appear to not learn from mistakes \cite{dickson18}. This affects the public view of the company in terms of how they value their social responsibility to report possible crime. The appointment of Comyn could be seen as endorsing his previous behaviour. \\
\hline
We get things done & Fix problems quickly, make timely decisions and focus on great outcomes & Failed to assess possible money laundering and terrorism financing risks of their deposit machines in 2017. Also failed to report on transactions more than \$10,000 and comply with requirements to monitor these transactions. Did not monitor known possible cases of money laundering \cite{AUSTRAC18}. & CBA paid a \$700 million penalty directly due to these breaches \cite{AUSTRAC18}. The bank was seen as not exercising the necessary measures to detect and fix money laundering and terrorism financing problems. This could be seen as not fixing problems in a timely manner, or focusing on great outcomes. \\
\hline
\end{longtable}

\section{Reliance on Information Systems and Respective Governance}
%5.	Given Commonwealth bank reliance on Information Systems (customer security, money transfers etc), to what extent does the Commonwealth Bank's Code of Conduct and Ethics specifically include IS Governance? 

Specific mentions of the Information Systems (IS) Governance in CBA's Code of Conduct are rare. It is stated that a Key Group Policy is `Information Security`, however no indication for further reading is provided. There is a Privacy Policy that details how customer information is collected and managed in general. However, does not encompass the scope of IS such as online banking. The Code of Conduct explains that the bank expects to protect the integrity of the financial system, which may include IS as well \cite{code-of-conduct}.
\\

%How would Commonwealth Bank's Code of Conduct and Ethics be improved in relation to IT Governance? 
The Code of Conduct could be improved by creating a key Group Policy specific to IS. This is because Values Expectations and the key group policies outline the expected standards \cite{code-of-conduct}. The Group Policy could then refer to related policies and documents. An example of a related policy could be an Online Banking and Transactions policy that specifically outlines responsibilities in online banking activities. 
\\

%How could increasing the level of IT Governance to Commonwealth Bank's Code of Conduct and Ethics add value (both tangible and intangible) and reduce risk (in terms of operation risk and systematic risk) to the bank? Briefly explain your reasoning and include illustrative examples to reinforce your point of view. (200 words, 25%)
This increase in IS Governance could make the bank more directly accountable in terms of what they can and cannot do with IS. It reduces the risk of unintentional IS misuse by essentially removes any grey area that may exist in the existing ethical standards. This reduces operation risk in terms of employee errors and system misuse. As it reduces operation risk, it adds value to the firm. The Online Banking and Transactions policy, could outline the minimum security requirements for transactions are secured online. This leaves no grey area for how transactions should be implemented.

%Response is very insightful. Very well written and supported argument. Insightful discussion on how adding IT Governance to CBA Code of Conduct and Ethics could have helped CBA in terms of value and risk.
%governance and controls -> information security


\section{Corporate Social Responsibility}
%6.	Define corporate social responsibility (CSR) and how CSR relates to ethics? What are Commonwealth Bank's view and actions on CSR? How does it relate to their Code of Ethics? (250 words, 15%)
%Answers the question with a high level of detail and insightfulness. Gives excellent definitions of CSR and provides a excellent coverage of how it relates to ethics. Answer is insightful with a very high level of detail and research.  Gives extremely succinct and high level of detail on definitions of CBA's view  and actions relating to CSR and details very clearly how it relates to their code of ethics.
Corporate social responsibility (CSR) is the responsibility of a corporation for the social and environmental impacts of their activities \cite{Chen20}. 


\section{Performance Against United Nations Global Compact Principles}
%7.	United Nations Global Compact is the world's largerst coporate sustainability initiative. Is the Commonwealth Bank a signatory to this initiative. Discuss how do you think Commonwealth Bank has been performing in relation to the Ten Principles. (150 words, 15%)
%Answers the questions with a extremely high level of detail and is very succint. Demonstrates insightful understanding of UN Global Compact and The Ten Principles' mission. Discusses in detial and provides insightful implications of CBA' performance in the Environment section of the Ten Principles. 


\begin{longtable}{|p{5cm}|p{10.5cm}|} 
\caption{CBA performance against UN Global Compact Principles}
\label{tab:breaches}
\hline
% longtable formatting of headers over multiple pages
\textbf{Principle} & \textbf{CBA Performance} \hline \endfirsthead

\multicolumn{2}{c}
        {Table \thetable\ continued from previous page} \\
        \hline
\textbf{Principle} & \textbf{CBA Performance} \hline \endhead
%
\multicolumn{2}{|r|}{{Continued on next page}} \\ \hline
\endfoot
%
\endlastfoot

% Table data
1. Businesses should support and research the protection of internationally proclaimed human rights & \\
\hline
2. Make sure that they are not complicit in human rights abuses & \\
\hline
3. Businesses should uphold the freedom of association and the effective recognition of the right to collective bargaining & \\
\hline
4. The elimination of all forms of forced and compulsory labour & \\
\hline
5. The effective abolition of child labour & \\
\hline
6. The elimination of discrimination in respect of employment and occupation.  & \\
\hline
7. Businesses should support a precautionary approach to environmental challenges & \\
\hline
8. Undertake initiatives to promote greater environmental responsibility & \\
\hline
9. Encourage the development and diffusion of environmentally friendly technologies & \\
\hline
10. Businesses should work against corruption in all its forms, including extortion and bribery. & \\
\hline
\end{longtable}


%Provide a list of references and is within word limits (+/-10%). Has low similarity on turnitin (<20%). (10%)
\bibliographystyle{ieeetr}
\bibliography{mybib}

\end{document}
